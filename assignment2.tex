\documentclass[a4paper]{article}
\usepackage{mathtools}
\title{Assignment 2: RSA and DH Algorithms}
\author{Abdullah Meraj \and Yazeed Alkhalaf \and Affan Mohammed \and Bara Allam}
\date{\today}
\begin{document}

\maketitle
\section*{Exercise 1: RSA Encryption}
\subsection*{1.1: Explain the RSA Encryption Algorithm}
\subsection*{1.2: Calculation of RSA Parameters}
\subsubsection*{1. Key Generation}
\begin{itemize}
    \item Ahmed chooses two prime numbers: p = 17 and q = 11.
    \item Calculate \(n,\ \phi(n) \):
          \[ n = p \times q = 17 \times 11 = 187 \]
          \[\phi(n) = (p - 1) \times (q - 1) = (17 - 1) \times (11 - 1)\]
          \[ = 16 \times 10 = 160\]

    \item Choose \( e = 13 \) for Ahmed’s public key.

    \item Calculate Ahmed’s private key \( d \):
          \[
              Here \ e = 13, \ \phi(n) = 160
          \]
          \[
              ed \equiv 1 (\mod\phi(n))
          \]
          \[
              \implies ed \mod \phi(n) = 1\mod\phi(n)
          \]
          \[
              \implies ed\mod\phi(n) = 1
          \]
          \[
              let \ ed = k \times \phi(n) + 1
          \]
          \[
              13d = k \times 160 + 1
          \]
          \[
              d = \frac{k \times 160 + 1}{13}
          \]
          For k = 1:
          \[
              d = \frac{1 \times 160 + 1}{13} = \frac{160 + 1}{13} = \frac{161}{13} = 12.38
          \]

          For k = 2:
          \[
              d = \frac{2\times 160 + 1}{13} = \frac{320 + 1}{13} = \frac{321}{13} = 24.69
          \]

          For k = 3:
          \[
              d = \frac{3\times 160 + 1}{13} = \frac{480 + 1}{13} = \frac{481}{13} = 37
          \]

          d = 37 \newline Ahmed’s private key is (37,187) \subsubsection*{2. Encryption}
          \textbf{Question: }Fahd wants to send a plaintext message: “SECRET” to Ahmed.
          He converts the plaintext into numeric representation using a predetermined
          mapping (A=1, B=2, C=3 and so on). Then, he encrypts the numeric representation
          of the message using Ahmed’s public key (e, n). \newline

          \textbf{Answer: } S = 19, E =
          5, C = 3, R = 18, T = 20 \ n = 187, e = 13 \
          \[ C = M^e \mod n \]
          \[ C = M^{13} \mod 187\]
          \[C_S = 19^{13} \mod 187 = 83\]
          \[C_E = 5^{13} \mod 187 = 37\]
          \[C_C = 3^{13} \mod 187 = 148\]
          \[C_R = 18^{13} \mod 187 = 35\]
          \[C_T = 20^{13} \mod 187 = 80\]
          "SECRET" is encrypted as "83 37 148 35 37 80"

          \subsubsection*{3. Decryption}
          \textbf{Question: }Ahmed receives another encrypted message from Fahd: “94 37 133 133 53” \newline

          \textbf{Answer: }
          Here d = 37, n = 187
          \[ M = C^d \mod n \]
          \[ M = C^{37} \mod 187 \]
          \[ M_94 = 94^{37}\mod 187 = 8 = "H" \]
          \[ M_37 = 37^{37}\mod 187 = 5 = "E" \]
          \[ M_{133} = 133^{37}\mod 187 = 12 = "L" \]
          \[ M_{53} = 53^{37}\mod 187 = 15 = "O" \]
          "94 37 133 133 53" is decrypted as "HELLO"
\end{itemize}
\end{document}